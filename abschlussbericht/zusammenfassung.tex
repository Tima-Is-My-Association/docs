\chapter{Zusammenfassung}
Das Projekt TIMA besteht inzwischen aus der Datenbank mit dazugehöriger API. Diese setzt für Autorisierung auf ein eigenes Protokoll. Um Nutzer für das Projekt zu gewinnen, setzen wir vor allem auf Spiele und Wettbewerbcharakter durch Ranglisten. Wir haben eine Applikation implementiert, die das Spiel Assoziationskette auf mobilen Geräten ermöglicht. Die Webseite von TIMA bietet dieses Spiel ebenfalls an und zusätzlich noch verschiedene andere Funktionen, wie die Statisk über Nutzer, Sprachen, Worte Assoziationen und die Rangliste der besten Assoziierer.
\newline

Momentan befinden sich über 3000 Worte mit mehr als 4000 Assoziationen in der Datenbank. Verteilt sind diese auf vier verschiedene Sprachen, Deutsch, Englisch, Spanisch und Farsi. Die am stärksten vertretene Sprache ist mit großem Abstand Deutsch. Dies rührt sicherlich daher, dass TIMA bisher keine große Verbreitung gefunden hat. Die bisherigen zwölf angemeldeten Nutzer stammen vor allem aus dem Bekanntenkreis der Programmierer. Für die Zukunft gibt es viele verschiedene Anwendungsmöglichkeiten für TIMA, die den Nutzerkreis deutlich erweitern können.