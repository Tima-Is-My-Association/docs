\chapter{Zusammenfassung}
TIMA ist eine Datenbank zum sammeln von Assoziationen. Den Hauptbestandteil bildet die Webseite bestehend aus Front- und Backend. Das Backend besteht aus der eigentlichen Datenbank und einer umfangreichen API. Diese setzt für Authentifizierung auf ein eigenes Protokoll. Um Nutzer für das Projekt zu gewinnen, setzen wir vor allem auf Spiele und Wettweberbcharakter durch Ranglisten. Zusätzlich zum Webfrontend haben wir eine App implementiert, die das Spiel Assoziationskette auf mobilen Geräten ermöglicht.

Momentan befinden sich über 3000 Worte mit mehr als 4000 Assoziationen in der Datenbank. Verteilt sind diese auf vier verschiedene Sprachen: Deutsch, Englisch, Persisch und Spanisch. Die am stärksten vertretene Sprache ist mit großem Abstand Deutsch. Dies rührt sicherlich daher, dass TIMA bisher keine große Verbreitung gefunden hat. Die bisherigen zwölf angemeldeten Nutzer stammen vor allem aus dem Bekanntenkreis der Programmierer. Für die Zukunft gibt es viele verschiedene Anwendungsmöglichkeiten für TIMA, die den Nutzerkreis deutlich erweitern können.