\chapter{Ausblick}\label{ch:ausblick}

Die bisher verrichtete Arbeit bietet sehr gute Vorraussetzungen für zukünftige Projekte, die entweder mit den Assoziationen arbeiten oder die Assoziationsdatenbank erweitern möchten. Durch strikte Trennung von Front- und Backend ist ein einfaches Austauschen beider Bestandteile jederzeit möglich.
Um die Datenbank weiter zu füllen, waren einige weitere Spiele beziehungsweise Funktionen für bestehende Spiele geplant, die jedoch aus Zeitgründen nicht weiter verfolgt wurden.

\paragraph{Assoziationskette}
Momentan fehlt eine Implementierung als App, in der dann ebenfalls auch kooperativ mit anderen menschlichen Gegnern gespielt werden kann. Eine interessante Spielart könnte dabei sein, zufälligen einen Mitspielern zugewiesen zu bekommen. Dies kann auch ein Computergegner sein. Nach Beendigung des Spiels den Spieler zu fragen, ob er mit einem Menschen oder dem Computer gespielt hat, ist eine interessante Abwandlung des Turingtests, der auch über die Qualität unserer Assoziationsdatenbank entscheiden kann.

\paragraph{Familienduell}
Das Spiel Familienduell ist momentan nicht implementiert, die Planung aber abgeschlossen, bietet jedoch großen Unterhaltungswert. Der Name soll hierbei nur die Ausrichtung des Spieles näher bringen,
denn wie bei der
Fernsehserie\footnote{\url{https://de.wikipedia.org/wiki/Familien-Duell}},
werden dem Nutzer verschiedene verdeckte Antworten auf eine Frage gezeigt
und für jede richtig gegebene Antwort erhält der Spieler Punkte.
Die Fragen sind jedoch im Unterschied zum Fernsehen, ausschließlich die meist
genannten Assoziationen zu einem bestimmten Wort. Da alle korrekten Antworten auf TIMA nachgeschaut werden können, sollte zusätzlich eine
Zeitbegrenzung für das Eingeben implementiert werden, um so einen zusätzlichen Anreiz zu schaffen selbständig zu antworten. Je nach Schwierigkeitsgrad kann ein Zeitbonus für korrekte Assoziationen
gegeben werden und ein Malus bei falschen Antworten. Die Einwirkung auf das
Spielvergnügen müsste entsprechend getestet werden.

Jede gegebene Antwort sollte für das Füllen der Datenbank verwendet werden, obgleich
es eine gesuchte Lösung war oder nicht. Punkte sollte ein Spieler jedoch nur
für richtige Lösungen und einen Bonus für alle Lösungen erhalten.

\paragraph{Andere Sprachen}
Wünschenswert wäre auch eine bessere Umsetzung aller Komponenten in andere Sprachen.
Zum einen sollen weitere Sprachen für die Assoziationen hinzugefügt werden, so dass eine noch größere Nutzerschaft angesprochen werden kann. Derzeit werden hier die Sprachen Deutsch, Englisch, Persisch und Spanisch unterstützt.

Zum anderen soll das Webfrontend und die Applikation mehrsprachig sein. Hier ist momentan lediglich das Webfrontend in Deutsch und Englisch verfügbar.

\paragraph{Analsye der Daten}
Mit den vorhandenen Daten lassen sich viele Analysen durchführen. Zum Beispiel wäre ein Vergleich zwischen Kookurrenzen und Assoziationen möglich. Doch nicht nur im Bereich der automatischen Sprachverarbeitung lassen sich sicherlich interessante Beobachtungen feststellen. Wie sich der Hintergrund des Nutzers auf seine Assoziationen auswirken, ist ein spannender sozialwissenschaftlicher Aspekt.

Wir wünschen uns in der Zukunft viele Projekte, die TIMA unterstützen, weiter aufbauen und natürlich nutzen.
