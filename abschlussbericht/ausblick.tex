\chapter{Ausblick}

Die bisher verrichtete Arbeit bietet sehr gute Vorraussetzungen für zukünftige Projekte, die entweder mit den Assoziationen arbeiten, oder die Assoziationsdatenbank erweitern möchten. Durch strike Trennung von Front- und Backend ist einfaches Austauschen beider Bestandteile jederzeit möglich.
Um die Datenbank weiter zu füllen, waren einige weitere Spiele beziehungsweise Funktionen für bestehende Spiele geplant, die jedoch aus Zeitgründen nicht weiter verfolgt wurden.

\paragraph{Assoziationskette}
Momentan fehlt der Assoziationskette die Möglichkeit, kooperativ mit anderen menschlichen Gegnern zu spielen. Diese könnte in Zukunft implementiert werden. Eine interessante Spielart könnte dabei sein, zufälligen Mitspielern zugewiesen zu werden. Dies könnte auch ein Computergegner. Nach dem Spiel mit einem zufälligen Spieler den Spieler zu fragen, ob er mit einem Menschen oder einem Computer gespielt hat, ist eine interessante Abwandlung des Turingtests, der auch über die Qualität unserer Assoziationsdatenbank entscheiden kann.

\paragraph{Familienduell}
Das Spiel Familienduell ist momentan nicht implementiert, bietet jedoch großen Unterhaltungswert. Der Name soll hierbei nur die Ausrichtung des Spieles näher bringen,
denn wie bei der
Fernsehserie\footnote{siehe \url{https://de.wikipedia.org/wiki/Familien-Duell}},
werden dem Benutzer verschiedene verdeckte Antworten auf eine Frage gezeigt
und für jede richtig gegebene Antwort erhält der Spieler Punkte.
Die Fragen sind jedoch im Unterschied zum Fernsehen, ausschließlich die meist
genannten Assoziationen zu einem bestimmten Wort. Zusätzlich sollte eine
Zeitbegrenzung eingehalten werden, da ja alle Korrekten Antworten auf TIMA
nachgeschaut werden können und so einem möglichen Betrug entgegen gewirkt werden
kann. Je nach Schwierigkeitsgrad kann ein Zeitbonus für korrekte Assoziationen
gegeben werden und ein Malus bei falschen Antworten. Die Einwirkung auf das
Spielvergnügen müsste entsprechend getestet werden.

Jede gegebene Antwort sollte für das füllen der Datenbank verwendet werden, obgleich
es eine gesuchte Lösung war oder nicht. Punkte sollte ein Spieler jedoch nur
für richtige Lösungen und einen Bonus wenn er alle Lösungen findet erhalten.

\paragraph{Andere Sprachen}
Wünschenswert wäre auch eine bessere Umsetzung aller Komponenten in andere Sprachen. Zum aktuellen Zeitpunkt ist eine Rechtschreibprüfung nur für Deutsch eingefügt und sowohl die Webseite als auch die App sind nur auf Deutsch und Englisch verfügbar.

\paragraph{Analsye der Daten}
Mit den vorhandenen Daten lassen sich sicherlich auch viele Analsyen durchführen. Zum Beispiel wäre ein Vergleich zwischen Kookurrenzen und Assoziationen möglich. Doch nicht nur im Bereich der automatischen Sprachverarbeitung lassen sich sicherlich interessante Beobachtungen feststellen. Wie sich der Hintergrund des Nutzers auf seine Assoziationen auswirken, ist ein spannender Faktor.
\newline

Wir wünschen uns in der Zukunft viele Projekte, die TIMA unterstützen, weiter aufbauen und natürlich nutzen.