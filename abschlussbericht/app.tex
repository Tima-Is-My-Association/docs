\chapter{Applikationen}
Die Applikationen sollen die Verwendung vom TIMA ohne Webbrowser, für möglichst
viele Endnutzer möglich machen.  Besonders für Mobile Geräte soll die Nutzung
vereinfacht werden.

Als erste Variante wurden grundlegende Funktionen der
Webseite nachgebaut.  Dies diente auch der Entwicklung der API, um etwaige
Fehler im Protokoll oder Verbesserungen dessen aufzuzeigen und zu beheben.

\section{Framework}
Als Framework wurde sich für Qt5 entschieden Aufgrund der weitreichenden
Unterstützung des Frameworks von Endgeräten.

Es wurde zudem Java in betracht bezogen, da dies vorrangig auf Mobilen Geräten
Verwendung findet. Allerdings wurde sich aus Sympathiegründen des Entwicklers
dagegen entschieden.

\section{Aufbau}
Die innere Logik wird durch einen Zustandsautomaten dargestellt um
Mehrfachanfragen zu vermeiden und eine einfache Fehlerkorrektur zu ermöglichen.
In Abbildung \ref{fig:uml_automata} wird der Zusammenhang der einzelnen
Zustände angezeigt. Das wechseln der Zustände wir rein über die Qt eigenen
Signale geregelt.
\begin{figure}[!h]
	\centering
	\includegraphics[width=\textwidth]{../UML/app_automata.png}
	\caption{UML State Diagramm des Applikationszustandsautomaten}
	\label{fig:uml_automata}
\end{figure}

\section{Sicherheit}
Die Sicherheit hat bei der Entwicklung eine große Rolle gespielt.  Auch wenn
das Projekt nicht unbedingt einer größeren Sicherheitsvorkehrung als SSL
bedürfte, wurde sich für die Absicherung von Teilen der API durch
unauthorisierte Dritte Abgesichert. Dies dient dazu, lediglich von TIMA
akzeptierte Applikationen Schreibrechte für die Datenbank zu geben.  Das Auslesen
der Informationen bliebt jedoch davon unangetastet, sofern es sich nicht um Benutzer spezifische Daten handelt, und ist nach wie vor für
jeden offen.

\paragraph{Sicherheitsplanung}
Zuerst wurde eine Verschlüsselung mit PGP geplant, um eine
vollständige \glqq Ende zu Ende\grqq~Verbindung herzustellen und auch \glqq Man
in the Middle\grqq~Atacken auszuschließen. Doch aufgrund von erhöhtem
Programmieraufwand wurde hierauf verzichtet.

Als zweites wurde eine Implementierung vom OAuth2-Standard versucht.
Die Anforderungen bestanden hierbei an einem geheimen Schlüssel in der
Applikation, dem Ausschluss einer Replay-Attacke und dem Verwenden von
User eigenen Passwörtern.

Im Standard sind 4 verschiedene \glqq Authorization Grants\grqq~
definiert\footnote{siehe \url{https://tools.ietf.org/html/rfc6749\#section-1.3}}
jedoch erfüllt keine davon jede Anforderung an das gewünschte Protokoll.

Zuletzt wurde sich für eine Eigenentwicklung entschieden um sowohl den Arbeitsaufwand
klein zu halten, als auch alle Anforderungen ausreichend zu erfüllen.
Eine kleine Übersicht ist in Abbildung \ref{fig:uml} zu sehen.

Die Übertragung erfolgt mit JSON Objekten um das Parsen zu erleichtern und
einen üblichen Standard zu verwenden. Der Rechenaufwand des Servers wurde
zudem versucht so gering wie möglich zu halten, um mögliche DDOS Angriffe
abzuschwächen.

\subsection{Authentisierung}
Eine Authentisierung ist wie folgt aufgebaut.
Der Client sendet eine Authentisierungsanfrage an den Server mit seinem
Benutzernamen und einer Applikations-Identifikationsnummer. Der Nutzername
dient dem Server zur Zuordnung zum Benutzerprofil und die Identifikationsnummer
regelt die Zugehörigkeit der verwendeten Applikation. Es werden nur registrierte
Applikationen akzeptiert.

Bei korrekten Informationen sendet der Server seine aktuelle Uhrzeit zurück
und speichert diese unter dem angegebenen Nutzerprofil ab. Dieser Zeitstempel
wird für die Signierung der weiteren Pakete verwendet und um nicht wiederkehrende
Signaturen zu erzwingen.

Im nächsten Schritt erstellt der Client das eigentliche Authentisierungs Paket mit
seinem Benutzernamen, seinem Passwort, der Identifikationsnummer, dem Zeitstempel
und einem Token als Signatur, welcher aus dem Hash des Zeitstempels und dem
konkatenierten Applikations-Geheimnis besteht. Als Hashmethode wurde hierbei
SHA-512 verwendet.

Nachdem der Server alle Angaben überprüft hat und die Signatur korrekt
verifizieren konnte, erhält der Client eine laufende Paketnummer, einen für ihn
geltende Benutzer-Identifikationsnummer und einen Usertoken.

Für jede weitere API Abfrage muss mindestens die Benutzer-Identifikationsnummer
und eine Signatur bestehend aus dem SHA-512 Hash des Usertoken konkateniert mit
der Paketnummer verschickt werden. Nach jeder erfolgreichen API Anfrage welche
die Paketnummer und die Signatur verlangt, wird die Paketnummer um eins erhöht.
Aus Kompatibilitätsgründen läuft die Paketnummer bis maximal 2147483646 und
fängt danach wieder bei 0 an.


\section{Spiele}\label{sec:spiele}
Um mehr Nutzer zu erzielen sind verschiedene Spiele geplant worden, die jedoch
aus Zeitgründen nicht in den geplanten Zeitrahmen passten.
Als Grundlage wird jedoch in den meißten Fällen schon eine gut gefüllte Datenbank
vorausgesetzt. In den folgenden Abschnitten, werden kurz die Spiele
Assoziationskette und Familienduell in ihrer Funktion beschrieben und der Nutzen
für die Datenbank näher erläutert.

\paragraph{Assoziationskette}
Die Spieler assoziieren nacheinander auf die jeweilige Assoziation ihres
Vorgängers. Dadurch Bildet sich eine Kette mehrerer Assoziationen bis
zu bestimmten Abbruchbedingungen. Ein mögliches Ende stellt das assoziieren
eines schonmal verwendeten Wortes dar und die damit verbundene Schließung eines
Kreises in der Kette. Das Spiel ist ebenfalls beendet, sollte einem Spieler
keine weitere Assoziation zeitnah einfallen. Bei einem Computergegner wird
so lange wie möglich versucht die Abschlusskriterien nicht zu erfüllen, damit
der Spieler selbst in den meißten Fällen dafür verantwortlich ist.

Alle gegebenen Assoziationen des Spielers werden hierbei mit Punkten belohnt
und für die Datenbank verwendet.Die Gesammtlänge der Kette wird nochmal extra
als Bonus berechnet, hat jedoch keine Auswirkungen auf die Datensätze in der
Datenbank.

\paragraph{Familienduell}
Der Name soll hierbei nur die Ausrichtung des Spieles näher bringen,
denn wie bei der
Fernsehserie\footnote{siehe \url{https://de.wikipedia.org/wiki/Familien-Duell}},
werden dem Benutzer verschiedene verdeckte Antworten auf eine Frage gezeigt
und für jede richtig gegebene Antwort erhält der Spieler Punkte.
Die Fragen sind jedoch im Unterschied zum Fernsehen, ausschließlich die meist
genannten Assoziationen zu einem bestimmten Wort. Zusätzlich sollte eine
Zeitbegrenzung eingehalten werden, da ja alle Korrekten Antworten auf TIMA
nachgeschaut werden können und so einem möglichen Betrug entgegen gewirkt werden
kann. Je nach Schwierigkeitsgrad kann ein Zeitbonus für korrekte Assoziationen
gegeben werden und ein Malus bei falschen Antworten. Die Einwirkung auf das
Spielvergnügen müsste entsprechend getestet werden.

Jede gegebene Antwort wird für das füllen der Datenbank verwendet, obgleich
es eine gesuchte Lösung war oder nicht. Punkte erhält der Spieler jedoch nur
für richtige Lösungen und einen Bonus wenn er alle Lösungen findet.























