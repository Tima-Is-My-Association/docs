\chapter{TIMA als Citizen Science Projekt}
Im folgenden Kapitel wird der Rahmen von Citizen Science definiert und die Einordnung von TIMA darin. Anschließend erfolgt eine Erläuterung der Grundlagen von TIMA beginnend mit der Dateneingabe. Im weiteren Verlauf des Kapitels, folgt die Sicherung der Datenqualität und die Lizenz.

\section{Citizen Science}
Für Citizen Science gibt es keine klare Definition. Es gibt verschiedene Grenzen die um den Begriff Citizen Science gezogen werden. Unterschieden wird dabei z.B. zwischen Citizen Science und Crowdsourcing, jedoch sind die Übergänge meist fließend.

Als ein Maßstab der zur Abtrennung zwischen Citizen Science und Crowdsourcing benutzt wird M. Haklays Stufen der Beteiligung genutzt. Dabei werden vier Stufen unterschieden. Am unteren Ende der Skala steht dabei das Crowdsourcing auf Stufe eins, bei dem die Teilnehmer als Sensoren oder Freiwillige mitarbeiten. Am oberen Ende, auf Stufe vier, steht das „extreme“ Citizen Science, bei dem sowohl die Problemdefinition, als auch die Datensammlung und -auswertung von den Teilnehmern gemacht wird.\cite{wiggins2011conservation}
\\~\\
Eine weiter Abgrenzung erfolgt über eine Einordnung des Citizen Science Projektes selber in fünf verschiedene Typen:

\paragraph{Aktion (Action)} bezeichnet meist das Involieren von Teilnehmern in Lokale Anliegen, die nicht von Wissenschaftlern organisiert sind, aber oft eine beratende Funktion dabei einnehmen. Es handelt sich in der Regel um langfristige Bewegungen mit gesellschaftlichen Zielen.

\paragraph{Erhaltung (Conservation)} umfasst das Verwalten mit eindeutigen Lehrzielen. Diese sind meist jedoch nicht in einem leicht zugänglichen Format und stark abhängig von staatlichen Mitteln.

\paragraph{Die Untersuchung (Investigation)} beschäftigt sich mit dem Sammeln von Daten die für ein spezifisches Forschungsziel von Nöten sind. Die Projekte müssen in wissenschaftlichen Rahmenbedingungen geführt werden.

\paragraph{Virtuell (Virtual)} ist mit dem vorherigen Punkt „Untersuchung“ zu vergleichen. Der Unterschied besteht darin, das keine physikalische Datensammlung notwendig ist.

\paragraph{Lehre (Education)} hat zum Hauptziel, die Verbreitung von Wissen. Gültige Resultate haben hierbei weniger Wert, als der Lernprozess an sich.
 
Für uns muss ein Projekt, damit es als Citizen Science angesehen werden kann, zu mindestens einem der fünf Typen zugeordnet werden können. Des weiteren muss es in M. Haklays Stufen der Beteiligung mindestens auf Stufe zwei einsortiert werden können, da bei Stufe eins die Teilnehmer nicht ausreichend konstruktiv am Problem mitarbeiten.

\subsection{TIMA im Citizen Science Rahmen}
In einem Citizen Science Projekt sind weitere Problemstellungen zu beachten, als in einem gewöhnlichen Informatikprojekt. Die Nutzer sind keine bezahlten oder interessierten Angestellten, sondern gehören der breite Masse an. So nutzen wir die Intelligenz der Masse, die wichtigsten Assoziationen zu einem bestimmten Wort zu finden. Eine sinnvolle Assoziationsdatenbank lässt sich nur mit mithilfe einer großen Anzahl Nutzer erstellen.

In einem Citizen Science Projekt muss die \textbf{Datenqualität} überprüft und gesichert werden. Da fast jeder Nutzer ein Laie ist, gibt er eventuell nicht immer ideale Werte ein. Dadurch, dass Nutzer freie Werte eintragen können, ist das Projekt auch gegen Vandalismus besonders anfällig. Zuletzt muss bedacht werden, das die Nutzer in keiner Form bezahlt werden. Daher muss sich TIMA damit auseinander setzen, wie es Menschen die nötige \textbf{Motivation} liefert, etwas zum Projekt beizutragen. Dabei fokussieren wir uns auf Spiele, um einen Anreiz zum Mitwirken zu geben.

Wir ordnen TIMA dem Typ „Virtuell“ zu und setzen es auf Stufe zwei von M. Haklays Stufen der Beteiligung.

\section{Grundlegende Dateneingabe}
Die Dateneingabe erfolgt bei TIMA im einfachsten Fall entweder über die Webseite oder die bereits entstanden Applikationen. Dabei werden nacheinander, dem Nutzer Wörter gegeben, zu denen er eine Assoziation eingeben soll. Der dabei gewählte Algorithmus zur Wortauswahl wird im nächsten Abschnitt beschrieben.

\subsection{Wortauswahlalgorithmus}\label{subsec:Wortauswahlalgorithmus}
In diesem Abschnitt wird die Funktionsweise des Wortauswahlalgorithmus genauer beschrieben. In der Datenbank ist es wünschenswert, dass jedes Wort mindestens eine Assoziation hat und zusätzlich ein ungefähres Gleichgewicht an Assoziationen verteilt zwischen allen Worten herrscht. Der Wortauswahlalgorithmus ist zuständig für das Auswählen der Worte, für die eine Assoziation gegeben werden soll, damit diese Eigenschaften erreicht werden.

Daraus ergeben sich die folgenden Anforderungen an den Algorithmus:

\begin{itemize}
	\item Jeder Nutzer soll zu möglichst vielen verschiedenen Worten eine Assoziation geben.
	\item Wörter, die wenig assoziiert wurden, entweder insgesamt oder von einem einzelnen Nutzer, sollten für diesen Nutzer bevorzugt werden.
	\item Wörter sollen ausgeschlossen werden können, um z.B. zu verhindern, dass das selbe Wort mehrmals hintereinander kommt.
\end{itemize}

Der Algorithmus, der in TIMA umgesetzt ist und diese Anforderungen erfüllt, ist in \hyperref[lst:wortselect]{Listing \ref*{lst:wortselect}} aufgeführt. Zunächst werden alle Wörter der entsprechenden Sprache ausgewählt, davon werden anschließend alle Wörter, die nicht vorkommen sollen gefiltert. Danach wird in Abhängigkeit vom Status des Nutzers, die Liste der möglichen Wörter sortiert und auf 15 Wörter beschränkt, aus denen zufällig ein Wort gewählt wird.

\medskip
\begin{lstlisting}[basicstyle=\ttfamily,
backgroundcolor=\color{lightgray},
showspaces=false,
showstringspaces=false,
showtabs=false,
columns=fixed,
frame=lines,
numbers=left,
numbersep=5pt,
breaklines=true,
captionpos=b,
label=lst:wortselect,
caption=Wortauswahlalgorithmus]
w = [alle Worter einer Sprache]
w = w - [alle auszuschließenden Wörter]
if ( Annonymer Nutzer )
  w = [15 Wörter mit niedrigstem Häufigkeit aus w]
else
  w = w - [alle auszuschließenden Wörter des Nutzers]
  w = [15 Wörter mit niedrigstem Auftreten in der AssociationHistory des Buntzers1 aus w]
return [zufälliges Wort aus w]
\end{lstlisting}

\subsection{ExcludeWords}\label{subsec:excludewords}
Falls einem Nutzer zu einem Wort keine Assoziation einfällt, hat er die
Möglichkeit das Wort zu überspringen. Damit der Nutzer nicht in kurzer Zeit
wiederholt danach gefragt wird (vgl.
\hyperref[subsec:Wortauswahlalgorithmus]
{Abschnitt \ref*{subsec:Wortauswahlalgorithmus}}), wird das Wort auf eine
Ausschlussliste gesetzt. Einträge die älter als 7 Tage sind, werden
automatisch von der Liste gelöscht. Danach wird der Nutzer erneut nach
seiner Assoziation zu diesem Wort gefragt.

\subsection{Punkte}\label{subsec:punkte}
Um einen grundlegenden Anreiz für die  Mithilfe an TIMA zu geben, wird für jede Assoziation Punkte vergeben. Die Vergabe der Punkte erfolgt dabei nach \hyperref[eq:score]{Formel \ref*{eq:score}}. $ A_w $ steht dabei für die durchschnittliche Häufigkeit des Wortes, $ A_a $ die durchschnittliche Häufigkeit der Assoziation, $ x $ für die Häufigkeit des Wortes und $ y $ für die Häufigkeit der Assoziation.

\begin{equation}\label{eq:score}
p = \frac{A_w}{x} * \frac{y}{A_a} * 2
\end{equation}

\section{Spiele}\label{sec:games}
Darüber hinaus benutzen wir einen spielerischen Ansatz für das Eintragen von Assoziationen, um Menschen zur Mitarbeit bei TIMA zu motivieren. Derzeit ist lediglich das Spiel Assoziationskette umgesetzt, welches im nächsten Abschnitt ausführlich erläutert wird. Die Planung für das zweite Spiel Familienduell (vgl. \hyperref[ch:ausblick]{Kapitel \ref*{ch:ausblick}}) ist abgeschlossen, allerdings aus Zeitgründen noch nicht umgesetzt.

\subsection{Assoziationskette}
Dieses Spiel sollte für einen Nutzer deutlich interessanter sein, im Gegensatz zur Standardeingabe. Voraussetzung für dieses Spiel ist jedoch eine bereits gefüllte
Assoziationsdatenbank. Das Spiel Assoziationskette funktioniert folgendermaßen:

Die Spieler assoziieren nacheinander auf die jeweilige Assoziation ihres
Gegenübers. Dadurch bildet sich eine Kette mehrerer Assoziationen bis
zu bestimmten Abbruchbedingungen. Ein mögliches Ende stellt das Assoziieren
eines bereits verwendeten Wortes dar und die damit verbundene Schließung eines
Kreises in der Kette. Das Spiel ist ebenfalls beendet, sollte einem Spieler
keine weitere Assoziation zeitnah einfallen.

Das Spiel Assoziationskette steht derzeit nur als Variante mit einem Computergegner auf der Webseite zur Verfügung. Der Computer wählt dabei seine Assoziation zufällig aus den Assoziationen der Datenbank zu dem gegebenen Wort aus, wobei sichergestellt wird, das er keinen Kreis bildet.

Alle gegebenen Assoziationen des Spielers werden hierbei mit Punkten belohnt
und für die Datenbank verwendet, die Punkte berechnen sich dabei wie im \hyperref[subsec:punkte]{Abschnitt \ref*{subsec:punkte}} beschrieben. Die Gesamtlänge der Kette wird zusätzlich Bewertet und dem Sieger gutgeschrieben.

\section{Newsletter}\label{subsec:newsletter}
Ein weiteres Feature über das TIMA verfügt, ist der Newsletter. Falls ein
Nutzer möchte, kann er auf der Webseite Wörter auswählen, die ihn
interessieren und wöchentlich darüber einen Newsletter erhalten, in dem zu
jedem Wort die Assoziationen mit deren Häufigkeit enthalten sind. Durch die
regelmäßig Erinnerung könnte sich ein Nutzer motiviert fühlen, erneut an einem
Spiel für TIMA teilzunehmen. Dies erfordert eine vorherige Eintragung der E-Mail-Adresse, die für die Anmeldung nicht erforderlich ist.

\section{Datenqualität}
Um die Datenqualität  in der Datenbank zu sichern, setzen wir auf drei Dinge:

\begin{enumerate}
	\item Rechtschreibprüfung
	\item Mengenrelevanz
	\item Nutzermanagement
\end{enumerate}

\subsubsection{Rechtschreibung}
Die naheliegendste Überprüfung um die Datenqualität zu gewährleisten, ist eine
Rechtschreibkontrolle. Dabei wird nur auf Korrektheit eines Wortes geachtet und dem Nutzer werden keine Vorschläge zur Rechtschreibung unterbreitet.

Ein eingegebenes Wort wird zuerst mit der Datenbank abgeglichen, um zu schauen ob das Wort bereits bekannt ist. Wenn dies der Fall ist wird es als richtig angesehen ansonsten wird das Wort mit Wiktionary\footnote{https://www.wiktionary.org/} abgeglichen.

Sollte das Wort nicht gefunden werden wird der Nutzer darauf hingewiesen, damit er ggf. Änderungen vornehmen kann.

\subsubsection{Mengenrelevanz}
Mit einer Rechtschreibprüfung lässt sich natürlich nicht verhindern, dass ein
Nutzer falsche oder irrelevante Daten eingibt. Ob eine Assoziation falsch ist,
kann man natürlich von außerhalb nicht entscheiden. Jeder Mensch hat eigene
Assoziationen zu bestimmten Worten.

Wir betrachten eine Assoziation als relevant, wenn sie im Allgemeinen
nachvollziehbar ist. Gibt ein Nutzer eine Assoziation ein, die diesem
Anspruch nicht genügt, ob mutwillig oder nicht, kann man davon ausgehen, dass
andere Menschen diese Assoziation nicht geben. In der Datenbank wird eine
irrelevante Assoziation also statistisch vernachlässigbar.

\subsubsection{Nutzermanagement}
Ein weiteres mächtiges Werkzeug zur Vermeidung von Vandalismus ist die
Nutzerkontrolle. In der Datenbank werden alle Assoziationen gespeichert, die
ein Nutzer jemals gegeben hat. Sollte ein Nutzer auffällig werden, zum
Beispiel, weil er innerhalb sehr kurzer Zeit sehr viele Punkte in Spielen
erreicht, kann er überprüft werden. Sollte bei einer solchen Überprüfung
auffallen, dass seine Assoziationen sehr oft nicht dem allgemeinen Verständnis
entsprechen, können diese Eintragungen aus der Datenbank gelöscht werden.
Dieser Vorgang ist momentan nicht automatisiert.

\section{Lizenz}
Wir haben uns dazu entschieden, die Wörter und ihre Assoziationen unter die Creative Commons Attribution 4.0 International\footnote{http://creativecommons.org/licenses/by/4.0/} Lizenz zu stellen, damit jeder die Möglichkeit hat die Daten frei zu verwenden.

Den Quellcode hingegen haben wir unter eine angepasste LGPLv3\footnote{https://github.com/Tima-Is-My-Association/TIMA/blob/master/LICENSE} gesetzt. Zusätzlich wurden Einschränkungen gemacht, dass TIMA nicht in Bereichen eingesetzt werden darf, bei denen Menschenleben in irgendeiner Weise in Gefahr gebracht werden können.
