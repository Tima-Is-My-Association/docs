\chapter{Einleitung}

In Zeiten von schnellen Prozessoren und riesigen Speichermedien sind mithilfe von Text Mining und automatischer Sprachverarbeitung eine Vielzahl von Datenbanken entstanden, die ganze Sprachen aufgrund von Satzbau, Wortkookurrenzen und Wortarten analysieren und speichern. In dieser Menge der Daten fehlt jedoch eine sehr wichtige Eigenschafft von Worten - die Assoziation. Bisher war es nicht möglich, diese bedeutende menschliche Fähigkeit maschinell zu simulieren. TIMA, rekursives Palindrom für TIMA is my association, oder auf Deutsch: TIMA ist meine Assoziation, setzt es sich zum Ziel eine Datenbank zu schaffen, bei der diese Verbindungen zwischen Worten abgerufen werden können. Da wie beschrieben bisher keine automatische Methode dazu existiert, setzt TIMA auf eine Menge freiwilliger Nutzer, die ihre Assoziationen zu Worten eingeben.

Um eine relevante Menge an Daten sammeln zu können, wird TIMA als Citizen Science Projekt aufgezogen. Ein derartiges Projekt hat eine Reihe besonderer Ansprüche und um ihnen gerecht zu werden, sind eine ganze Reihe Vorkehrungen zu treffen, die in dieser Arbeit betrachtet werden sollen.

Zuerst wird näher betrachtet, warum das Erstellen einer Assoziationsdatenbank überhaupt sinnvoll ist und welche Funktionen für Nutzer ansprechend wären. Danach werden einige Gedanken zur Gestaltung  als Citizen Science Projekt geäußert. Besonders wird dabei auf den Schutz der gesammelten Daten eingegangen und der wichtigen Frage, wie man Nutzer motiviert, am Projekt teilzunehmen. Nach den technischen Details zur Implementierung unserer Bestandteile, der Datenbank, der API, der Webseite und einer ersten App, die die Möglichkeiten der Datenbank anschaulich demonstriert, folgt ein Abschnitt zum Ausblick auf zukünftige Projekte, die entweder zur Datenbank beitragen können oder sie nutzen.



\section{Motivation}
Die Möglichkeiten einer Assoziationsdatenbank sind vermutlich in erster Linie in Bereichen der automatischen Sprachverarbeitung angesiedelt, dort jedoch beinahe in jedem Teilgebiet nutzbar.

Ein großes Problem aller Automaten ist ihre Reaktivität. Sie sind stets auf Schlüsselbegriffe angewiesen, die der Nutzer eingibt, beziehungsweise spricht.  So ist für Suchanfragen jeglicher Art, ob nun Internetsuche, Eingabe in das Navigationsgerät oder Sprachbefehle neuartiger Steuerungen für mobile Endgeräte, wie zum Beispiel Siri, sehr schwer die korrekte Reaktion zu liefern, wenn der Nutzer von fest programmierter Terminologie abweicht. Sucht ein Autofahrer statt einer Tankstelle nach Benzin, wird er eventuell keine Antwort bekommen, obwohl einem Menschen intuitv klar ist, wonach der Fahrer sucht. Eine Maschine kann diese Schlüsse jedoch nicht ziehen und daher müsste man als Programmierer jede einzelne dieser Möglichkeiten bedenken und implementieren. Selbst für ein Navigationsgerät mit relativ eingeschränktem Handlungsspielraum ist dies schon sehr aufwendig, für eine Internetsuchmaschine oder Anwendungen im Bereich des Computational Advertisings jedoch quasi unmöglich. Die Varianz an Suchbegriffen ist einfach zu groß. 

Die Problematik einer solchen Datenbank ist jedoch, dass sie sich nicht automatisch erstellen lässt. Schon per Definition ist eine Assoziation eine vom Menschen gezogene Verbindung zwischen zwei Sachverhalten. Über Kookurrenzen lassen sich über Umwege ähnliche Ergebnisse erzielen. %FIXME: Ref?
Echte Assoziationen, wie sie Menschen ziehen, werden jedoch nur ein Bruchteil der Ergebnisse darstellen. Will man schlechte Ergebnisse vermeiden, ist es unumgänglich, die Assoziationsdatenbank per Hand von Menschen füllen zu lassen. Dass dies auf gewöhnlichem Weg ein sehr großer, auch finanzieller, Aufwand wäre, zeigt sich alleine daran, dass es bisher keine derartige Datenbank gibt, obwohl ein Nutzen, vor allem im Bereich Internetwerbung, nicht von der Hand zu weisen ist. Daher wollen wir hier den Citizen Science Ansatz benutzen, um eine derartige Datenbank zu realisieren.

Ein weiterer Vorteil, das Projekt mit Citizen Science Ansatz zu bearbeiten, bietet die größere Streuung von Assoziationen. Wenn eine einzelner Nutzer eine Assoziation zu einem bestimmten Wort eingeben soll, wird diese sehr oft die gleiche, oder zumindest eine sehr ähnliche sein. Wenn eine große Menge Personen Assoziationen eingibt, werden die Datenbank mit einer größeren Auswahl von Zusammenhängen efüllt. Stammen diese unterschiedlichen Personen auch noch aus sehr differenzierten Hintergründen, lokal und mit verschiedenen Interessen, so werden die Assoziationen sehr vielfältig. Ein Elektrotechniker wird sicherlich mit dem Begriff Halbleiter etwas anderes verbinden als ein Grundschullehrer. Ein Jugendlicher, der in einer Dorf am Meer groß geworden ist, wird vermutlich einen anderen Bezug zu Fisch haben, als ein Gleichaltriger aus einer Gebirgsstadt.

In den nachfolgenden Kapiteln wird erklärt, wie wir TIMA als Citizen Science Projekt umgesetzt haben.
